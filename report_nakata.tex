\documentclass{jsarticle}
\usepackage{graphicx, listings, jlisting, xcolor}
\lstset{%
  language=Verilog,
  belowcaptionskip=1\baselineskip,
  breaklines=true,
  frame=L,
  xleftmargin=\parindent,
  showstringspaces=false,
  basicstyle=\footnotesize\ttfamily,
  keywordstyle=\bfseries\color{green!40!black},
  commentstyle=\itshape\color{purple!40!black},
  identifierstyle=\color{blue},
  stringstyle=\color{orange},
}
\title{情報実験第三 1.B}

\author{情報工学科15\_03602 柿沼 建太郎 \\ 情報工学科 15\_10588 中田 光}
\date{\today}
\begin{document}
\maketitle

\section*{各課題担当者}
各課題と担当者を表として以下に示す。
\begin{table}[h]
\begin{tabular}{|l|c|c|} \hline
課題番号/名前 & 柿沼 & 中田 \\ \hline \hline
解析1 &  & $\circ$ \\ \hline
解析2 & $\circ$ & \\ \hline
解析3 & $\circ$ & $\circ$ \\ \hline
解析4 & $\circ$ & \\ \hline
解析5 & $\circ$ & \\ \hline
シミュレーション1 & $\circ$ & $\circ$ \\ \hline
シミュレーション2 &  & $\circ$  \\ \hline
\end{tabular}
\end{table}

\section*{Verilog記述解析レポート(1,2,3,4,5)}

\subsection*{課題1. EX3がリセットされる時、PCの値が0x010に設定される仕組み}

まず、com\_rstはcom\_ctrが00の時に1となるワイヤー変数として宣言されている。 \\
('COM\_RSTはdef\_ex3.vで00と定義されている)

\begin{lstlisting}[caption=cpu\_ex3.v 33行目]
wire com_rst = (com_ctl == `COM_RST); /// reset (pc, sc, ar, 1-bit FFs)
\end{lstlisting}

つまり、com\_rst=1となる時について考えればよい。\\
次に、reg\_lciとして定義されたSCでは、出力する値をクリアするフラグとしてsc\_clr|com\_rstが指定されている。 \\
com\_rst=1の時はSCの出力scは0となる。

\begin{lstlisting}[caption=cpu\_ex3.v 73行目]
reg_lci #3 SC  (clk, ~com_stop, 3'b0, sc, 1'b0, sc_clr | com_rst, ~sc_clr); /// if(sc_clr == 0) sc ++;
\end{lstlisting}

scは3to8デコーダDET\_Tによって8ビットワイヤ変数tに変換される。\\
sc=0の時、t[k] = 0 (k=0,1,2...7)となる。

\begin{lstlisting}[caption=cpu\_ex3.v 94行目]
dec_3to8 DEC_T (sc, t, 1'b1); /// (t[k] == 1) implies sc = k;
\end{lstlisting}

t[k] = 0 (k=0,1,2...7)となる時、3ビットワイヤ変数bus\_ctlは000となる。

\begin{lstlisting}[caption=cpu\_ex3.v 217~235行目]
wire bus_ar = d[4] & t[4] |   /// BUN @ t[4] : pc <- ar;
                     d[5] & t[5]; /// BSA @ t[5] : pc <- ar;
wire bus_pc = ~r & t[0] & ~com_rst | /// fetch @ t[0] : ar <- pc; (com_rst = 0)
                     r & t[1] | /// interrupt @ t[1] : mem[ar] <- pc;
                     d[5] & t[4]; /// BSA @ t[4] : mem[ar] <- pc;
wire bus_dr = d[6] & t[6]; /// ISZ @ t[6] : mem[ar] <- dr;
wire bus_ac = d[3] & t[4] | /// STA @ t[4] : mem[ar] <- ac;
                     pt & ir[10]; /// OUT : outr <- ac[7:0]
wire bus_ir  = ~r & t[2]; /// fetch @ t[2] : ar <- ir[11:0];
wire bus_mem = ~r & t[1] | /// fetch @ t[1] : ir <- mem[ar];
                     ~d[7] & i15 & t[3] | /// indirect : ar <- mem[ar];
                     d[0] & t[4] | /// AND @ t[4] : dr <- mem[ar];
                     d[1] & t[4] | /// ADD @ t[4] : dr <- mem[ar];
                     d[2] & t[4] | /// LDA @ t[4] : dr <- mem[ar];
                     d[6] & t[4]; /// ISZ @ t[4] : dr <- mem[ar];

assign bus_ctl[0]  = bus_ar | bus_dr | bus_ir | bus_mem; /// b1 | b3 | b5 | b7
assign bus_ctl[1]  = bus_pc | bus_dr | bus_mem; /// b2 | b3 | b7
assign bus_ctl[2]  = bus_ac | bus_ir | bus_mem; /// b4 | b5 | b7
\end{lstlisting}

bus\_ctlが000の時BUSではb0が選択され、 \\
bus\_dataには'PROGRAM\_ENTRY\_POINTが出力される。  \\
('PROGRAM\_ENTRY\_POINTはdef\_ex3.vで0x010と定義されている。)

\begin{lstlisting}[caption=cpu\_ex3.v 89行目]
bus BUS (bus_ctl, `PROGRAM_ENTRY_POINT, {4'b0, ar}, {4'b0, pc}, dr, ac, ir, 16'b0, mem_data, bus_data);
\end{lstlisting}

最後に、lciレジスタPC	において、 \\
com\_rst = 1の時にbus\_dataがロードされ、出力pcがbus\_dataの値(0x010)になる。

\begin{lstlisting}[caption=cpu\_ex3.v 89行目]
reg_lci #12 PC (clk, ~com_stop, bus_data[11:0], pc, pc_ld | com_rst, pc_clr, pc_inr);
\end{lstlisting}


\subsection*{課題2. 命令フェッチサイクルの動作がVerilogコード上でどのように実現されているか}

Verilogでは条件実行論理式を'wire'あるいは'assign'文の右辺に記述することで、条件が変化するとワイヤーのつながれた先の回路に無待機で情報が伝播する。

\begin{lstlisting}[]
wire ワイヤー名 = 条件式;
assign ワイヤー名 = 条件式;
\end{lstlisting}
'reg\_lci'または'reg\_lci\_nxt'で宣言されたmoduleはLCIレジスタである。\\
そこに0を代入する際にはそのクリア信号を1にし、インクリメントする際にはインクリメント信号を1にする。

\begin{lstlisting}[caption=ARの場合]
assign ar_clr = r & t[0]; //r & t[0]がtrueのときかつその時に限りARをクリア
assign ar_inr = d[5] & t[4]; //この条件のときのみインクリメントする
\end{lstlisting}

ロードする際にはロード信号を1にすればよいが、ロード先がレジスタによって異なる。\\
AR, PC, DR, IR, OUTRは入力がbus\_dataに繋がれており、LCIのロード信号をオンにした上でbus\_ctlの値を操作することで間接的にロードする内容を操作する。

\begin{lstlisting}[caption=$\overline R \cdot T(0) \Rightarrow AR \leftarrow PC$の場合]
assign ar_ld = ~r & t[0] |
               ~r & t[2] |
               ~d[7] & i15 & t[3];
wire bus_pc = ~r & t[0] & ~com_rst |
              r & t[1] |
              d[5] & t[4];
\end{lstlisting}
ACについては、ALU内で演算結果を格納するためのレジスタなので、ac\_ldが1になったときはALUの計算結果をロードするようになる。\\
なお、ac\_ldは
\begin{lstlisting}
assign ac_ld = ac_and | ac_add | ac_dr | ac_inpr | ac_cmp | ac_shr | ac_shl;
\end{lstlisting}
となっているため、ALUから結果をロードする以外にロードの機会はなく、事実上ACロード機能はないことがわかる。

SCについては、入力先とロード信号の部分に3'b0と1'b0が与えられているので、ロード機能は使われない。

LCIではないD-Flip-Flopで実現されたレジスタは'reg\_dff'というmoduleで宣言される。INPR, I, E, R, S, IEN, FGI, FGO, IOT, IMSKがそれにあたる。\\
これらに対する代入は、各レジスタに対してxx\_nxtという名前のワイヤーに条件式を充てておくことで実現される。Iについてのみir[15]が割り当てられているが、これもirの中身を見れば条件式と実質同様であるとわかる。

命令フェッチサイクルは条件と代入文の連続から成り、代入文は以上の機能を以て実現される。

\subsection*{課題3. ADD, LDA, CIR, CILにおいて、aluモジュールがどのような動作をするか}
柿沼: 
\subsubsection*{ADDについて}
関連するVerilogコードを抜粋する。
\begin{lstlisting}
wire [16:0] o_add = (ac_add) ? ({1'b0, dr} + {1'b0, ac}) : 17'b0;
assign ac_nxt = ... | o_add[15:0] | ...;
assign e_nxt = (ac_dd) | o_add[16] : ...;
\end{lstlisting}

aluモジュールに繋がれた入力信号ac\_addが立っていると、drとacを17ビットとして加算されたものがo\_addに代入される。\\
ac\_xx信号が同時に2つ以上立たないと仮定すれば、ac\_nxtには加算結果の下位16bitがそのまま代入される。\\
また、同様にしてe\_nxtにo\_addの最上位ビットが代入される。\\
これによって、[E, AC] = [0, AC] + [0, DR]という計算が実現される。\\

\subsubsection*{LDAについて}
関連するVerilogコードを抜粋する。
\begin{lstlisting}
wire [16:0] o_dr = (ac_dr) ? (dr) : 16'b0;
assign ac_nxt = ... | o_dr | ...;
assign e_nxt = ...  : e;
\end{lstlisting}
aluモジュールに繋がれた入力信号ac\_drが立っていると、drの中身がそのままo\_drへ代入される。\\
ac\_nxtに同様にo\_drがそのまま代入され、e\_nxtについてはどの条件にも触れないため、維持される。\\
これによって、AC = DRという計算が実現される。

\subsubsection*{CIRについて}
関連するVerilogコードを抜粋する。
\begin{lstlisting}
wire [16:0] o_shr = (ac_shr) ? ({e, ac[15:1]}) : 16'b0;
assign ac_nxt = ... | o_shr | ...;
assign e_nxt = ...  : (ac_shr) ? ac[0] : ...;
\end{lstlisting}
aluモジュールに繋がれた入力信号ac\_shrが立っていると、[E, AC[15 : 1]がo\_shrへ代入される。\\
ac\_nxtに同様にo\_shrが代入され、e\_nxtにはacの最下位ビットが代入される。\\
これによって、[AC, E] = [E, AC]という計算が実現される。

\subsubsection*{CILについて}
関連するVerilogコードを抜粋する。
\begin{lstlisting}
wire [16:0] o_shl = (ac_shl) ? ({ac[14:0], e}) : 16'b0;
assign ac_nxt = ... | o_shl;
assign e_nxt = ...  : (ac_shl) ? ac[15] : ...;
\end{lstlisting}
aluモジュールに繋がれた入力信号ac\_shlが立っていると、[AC[14 : 0], E]がo\_shlへ代入される。\\
ac\_nxtに同様にo\_shlが代入され、e\_nxtにはacの最上位ビットが代入される。\\
これによって、[E, AC] = [AC, E]という計算が実現される。 \\
\\
\\
中田:
\subsubsection*{ADDについて}
まず,ADDの実行サイクル6でac\_addが1となる。 \\
\begin{lstlisting}[caption=cpu\_ex3.v 166行目]
assign ac_add  = d[1] & t[5]; /// ADD @ t[5] : ac <- ac + dr;
\end{lstlisting}
次に、aluにおいて、ac\_add=1であるため\{0,dr\}と\{0,ac\}が加算されたものがo\_addに代入される。そして、ac\_nxtにはo\_add[15:0]、e\_nxtにはo\_add[16]が出力される。(変更のあったレジスタ以外のレジスタo\_*は0であるため、変更のあったレジスタの値がac\_nxtに代入される。e\_nxtは条件文に沿った値が代入される。他の命令についても同様である。) \\
これによってAC+DRの加算が実現する。
\begin{lstlisting}[caption=cpu\_module.v ALU modelより]
wire   [16:0] o_add  = (ac_add)  ? ({1'b0, dr} + {1'b0, ac}) : 17'b0;
assign ac_nxt = o_and | o_add[15:0] | o_dr | o_inpr | o_cmp | o_shr | o_shl;
assign e_nxt =  (ac_add) ? o_add[16] :
                    (ac_shr) ? ac[0] :
                    (ac_shl) ? ac[15] :
                    (e_clr)  ? 1'b0 :
                    (e_cmp)  ? ~e : e;
\end{lstlisting}

\subsubsection*{LDAについて}
まず,LDAの実行サイクル6でac\_drが1となる。 \\
\begin{lstlisting}[caption=cpu\_ex3.v 167行目]
assign ac_dr   = d[2] & t[5];    /// LDA @ t[5] : ac <- dr;
\end{lstlisting}
次に、aluにおいて、ac\_dr=1であるため、drがo\_drに代入される。そして、ac\_nxtにはo\_dr、e\_nxtにはeが出力される。 \\
これによりAC←DRが実現する。
\begin{lstlisting}[caption=cpu\_module.v ALU modelより]
wire   [15:0] o_dr   = (ac_dr)   ? (dr) : 16'b0;
assign ac_nxt = o_and | o_add[15:0] | o_dr | o_inpr | o_cmp | o_shr | o_shl;
assign e_nxt =  (ac_add) ? o_add[16] :
                    (ac_shr) ? ac[0] :
                    (ac_shl) ? ac[15] :
                    (e_clr)  ? 1'b0 :
                    (e_cmp)  ? ~e : e;
\end{lstlisting}

\subsubsection*{CIRについて}
まず,CIRの実行サイクル4でac\_shrが1となる。 \\
\begin{lstlisting}[caption=cpu\_ex3.v 170行目]
assign ac_shr  = rt & ir[7]; /// CIR : ac[14:0] <- ac[15:1], ac[15] <- e, e_nxt <- ac[0];
\end{lstlisting}
次に、aluにおいて、ac\_shr=1であるため、o\_shrに\{e,ac[15:1]\}が代入される。そして、ac\_nxtにはo\_shr、e\_nxtにはac[0]が出力される。 \\
これによりACの回転右シフトが実現する。
\begin{lstlisting}[caption=cpu\_module.v ALU modelより]
wire   [15:0] o_shr  = (ac_shr)  ? ({e, ac[15:1]}) : 16'b0;
assign ac_nxt = o_and | o_add[15:0] | o_dr | o_inpr | o_cmp | o_shr | o_shl;
assign e_nxt =  (ac_add) ? o_add[16] :
                    (ac_shr) ? ac[0] :
                    (ac_shl) ? ac[15] :
                    (e_clr)  ? 1'b0 :
                    (e_cmp)  ? ~e : e;
\end{lstlisting}

\subsubsection*{CILについて}
まず,CILの実行サイクル4でac\_shlが1となる。 \\
\begin{lstlisting}[caption=cpu\_ex3.v 171行目]
assign ac_shl  = rt & ir[6]; /// CIL : ac[15:1] <- ac[14:0], ac[0]  <- e, e_nxt <- ac[15];
\end{lstlisting}
次に、aluにおいて、ac\_shl=1であるため、o\_shlに\{ac[14:0],e\}が代入される。そして、ac\_nxtにはo\_shl、e\_nxtにはac[15]が出力される。 \\
これによりACの回転左シフトが実現する。
\begin{lstlisting}[caption=cpu\_module.v ALU modelより]
wire   [15:0] o_shl  = (ac_shl)  ? ({ac[14:0], e}) : 16'b0;
assign ac_nxt = o_and | o_add[15:0] | o_dr | o_inpr | o_cmp | o_shr | o_shl;
assign e_nxt =  (ac_add) ? o_add[16] :
                    (ac_shr) ? ac[0] :
                    (ac_shl) ? ac[15] :
                    (e_clr)  ? 1'b0 :
                    (e_cmp)  ? ~e : e;
\end{lstlisting}





\subsection*{課題4. FGIレジスタについて,fgi\_set, pt, ir, iot, fgiそれぞれの信号の組合せで出力値が決定する仕組み}
入出力のVerilogコードについて、関係のある個所のみ以下に抜粋する。
\begin{lstlisting}
reg_dff     #2  FGI  (clk, ~com_stop, fgi_nxt & {2{~com_rst}}  , fgi);  /// reset value = 00;
assign fgi_nxt[0] = (fgi_set[0]) ? 1         :  /// fgi_set[0] : fgi[0] <- 1
                    (pt & ir[11] & ~iot) ? 0 :  /// INP        : fgi[0] <- 0
                    fgi[0];                     /// unchanged
assign fgi_nxt[1] = (fgi_set[1]) ? 1         :  /// fgi_set[1] : fgi[1] <- 1
                    (pt & ir[11] &  iot) ? 0 :  /// INP        : fgi[1] <- 0
                    fgi[1];                     /// unchanged
edge_to_pulse #(4,0) FGP (clk, {fgi_bsy, fgo_bsy}, {fgi_set, fgo_set});
wire  pt = d[7] &  i15 & t[3]; /// @ t[3] : implies IO register-insn type

assign iot_nxt = (pt & ir[5]) ? 1            :  /// SIO : iot  <- 1
                 (pt & ir[4]) ? 0            :  /// PIO : iot  <- 0
                 iot;                           /// unchanged
\end{lstlisting}

FGIは2ビットのD-Flip-Flop型のレジスタで、0番はパラレル通信、1番はシリアル通信用となっている。\\
ptは間接アドレスフェッチサイクルのときに立つフラグで、ir[11]はワンホットコードのうちINP命令かどうかを示すbitである。\\
IOTは現在使用中なのがシリアルかパラレルかを示すフラグであり、SIO命令が来ると1に、PIO命令が来ると0になることから、0のときはパラレル通信で1のときはシリアル通信を表現する。\\
fgi\_setはであるようなフラグである。\\
このプログラムではedge\_to\_pulseをネガティブエッジパルス生成器としてインスタンス化しており、出力信号にfgi\_setが充ててあるためfgi\_setはfgi\_bsyの値が1から0になったクロックでのみ1を示す。\\

したがって、各FGIは、busy状態が解除されたら1, INP命令が来ておりかつ間接アドレスフェッチサイクルまで来ていてかつIOTによって自身が選択されていたら0に、そうでなければ維持される。

\subsection*{課題5. ARレジスタの出力信号がarとar\_nxtの2つある理由}
ex3のCPUの中でレジスタは大きく分ければLCIレジスタとDFFレジスタが使われており、それらはそれぞれ'reg\_lci'と'reg\_dff'という名前で定義されている。\\
しかしARレジスタに限り、'reg\_lci\_nxt'という名前のmoduleを使っている。\\
'reg\_lci\_nxt'は'reg\_lci'に加え、「まだ代入されていない値」を出力するという機能がついている。\\

一般的にレジスタへの代入は、レジスタへの入力信号に値を入れたまま待機し、クロックの立ち上がりが来た瞬間に出力にそれが反映される。\\
しかし、'reg\_lci\_nxt'ではまだ出力信号に来ていないような待機状態の値も出力するという機能がある。 この必要性について考える。\\

まず、メモリのロード先アドレスを示すワイヤー'mem\_addr'について次のような記述がある。
\begin{lstlisting}
wire [11:0]   mem_addr = (com_stop) ? com_addr : (mem_we) ? ar : ar_nxt;
\end{lstlisting}
'mem\_we'はメモリに対して書き込みを行うときに立つフラグであるため、メモリは書き込み時にはarを、読み込み時にはar\_nxtを使うことがわかる。\\
このことから、メモリを読むときだけはクロックの立ち上がりを待ってはいられない理由があると考えた。\\

ここで、間接アドレスフェッチサイクルからメモリ参照命令実行サイクルへ移行するタイミングについて考える。\\
まず、間接アドレスフェッチの立ち上がりでメモリからの読み出しが行われる(メモリは同期式なので立ち上がりの瞬間にしか出力の値は変更されない)。\\
次の立ち上がりでARへの書き込みが起きる。\\
すると、メモリ参照命令実行サイクルの始まりではARへの書き込みと読み出しが同時に行われるこことになる。\\
もしメモリからの読み出しにarを使っていたとすると、書き込み中の値を使用することになるため、値移行中の不定値を使用することになりメモリ読み出しが破綻する。\\
しかしar\_nxtを使うことにより、まだARの本来の出力に来ていない値を使用することができるため安全に読み出しができる。\\

ではすべての時にar\_nxtを使えばよいかというとそれでは問題が起きる場合がある。\\
BSA命令の実行サイクルを見てみると、
$AR \leftarrow AR + 1, M[AR] \leftarrow PC$
となっている。ARへの書き込みが起きている途中はar\_nxtの値が今度は不定となる。\\
したがって、書き込みのときには不定ではないarを使う必要がある。

\section*{Verilogシミュレーションレポート(1,2)}

\subsection*{1. 4つの課題プログラムそれぞれについて4つ以上の異なる入力に対する実行サイクル数とそれに関する考察}

\subsubsection*{倍制度乗算}
柿沼:
実行サイクル数
\begin{table}[h]
  \begin{tabular}{|l|l|l|l|l|} \hline
    入力($X \times Y$) & ステップ数(前回) & ステップ数(今回) & 実行サイクル数(今回) \\ \hline
    $11 \times 13$ & 90 & 93 & 445\\ \hline
    $0 \times 30$ & 114 & 117 & 564 \\ \hline
    $30 \times 0$ & 4 & 7 & 21\\ \hline
    $65535 \times 65535$ & 403 & 406 & 1985 \\ \hline
  \end{tabular}
\end{table}

\newpage

\subsubsection*{剰余算}
柿沼:
実行サイクル数
\begin{table}[h]
  \begin{tabular}{|l|l|l|l|l|} \hline
    入力($X \times Y$) & ステップ数(前回) & ステップ数(今回) & 実行サイクル数(今回) \\ \hline
    $30 \div 7$ & 340 & 343 & 1654 \\ \hline
    $0 \div 15$ & 336 & 339 & 1634 \\ \hline
    $65535 \div 1$ & 400 & 403 & 1954 \\ \hline
    $65535 \div 65535$ & 340 & 343 & 1654 \\ \hline
  \end{tabular}
\end{table}



\subsubsection*{16進$\rightarrow$ 10進}
柿沼:
実行サイクル数
\begin{table}[h]
  \begin{tabular}{|l|l|l|l|l|} \hline
    入力($X \times Y$) & ステップ数(前回) & ステップ数(今回) & 実行サイクル数(今回) \\ \hline
    $0F$ & 783 & 786 & 3832 \\ \hline
    $FFFF$ & 1957 & 1960 & 9583 \\ \hline
    $XX$ & 67 & 70 & 370 \\ \hline
    $ABCX$ & 157 & 160 & 820 \\ \hline
  \end{tabular}
\end{table}

\subsubsection*{素数計算}
柿沼:
実行サイクル数
\begin{table}[h]
  \begin{tabular}{|l|l|l|l|l|} \hline
    入力($X \times Y$) & ステップ数(前回) & ステップ数(今回) & 実行サイクル数(今回) \\ \hline
    $2$ & 27 & 30 & 129 \\ \hline
    $64$ & 61307 & 61310 & 297992 \\ \hline
    $255$ & 337101 & 337103\footnotemark[1] & 1638582\footnotemark[1] \\ \hline
    $65535$ & 245377052 & 245377054\footnotemark[1] & 1192716791\footnotemark[1] \\ \hline
  \end{tabular}
\end{table}
\footnotetext[1]{CPU\_MONITORINGを消したため、他のものとくらべ1少ない}

\subsubsection*{考察}
柿沼:
まずステップ数は前回の計測とくらべてすべて3ずつ増えている。
最後にHLTが来てから余計に3回HLT命令のところで止まっているのが確認されたため、そこが増えた回数である。
脚注[1]でも触れているが、デバッグ用のスイッチの切り替えにより増える数値が1~3まで変化することがわかっているので、おそらくVerilogシミュレーターのデバッグの仕様であると考えられるが、それ以上のことはわからなかった。

1ステップあたりの実行命令サイクル数を平均すると、約4.75回となった。
これは1つの命令あたりにかかるサイクルがおよそ4,5回という事実を意味し、実際、EX3の実行命令サイクルの表を見ると、多くの命令は5回のサイクルで終わる。平均回数が5回より少なくなっているのは出力などに要する割り込みサイクルであると考えられる。



\subsubsection*{倍制度乗算}
中田:
実行サイクル数
\begin{table}[h]
  \begin{tabular}{|l|l|l|l|l|} \hline
    入力 & ステップ数(前回) & ステップ数(今回) & 実行サイクル数(今回) \\ \hline
    $11 \times 13$ & 94 & 97 & 461 \\ \hline
    $0 \times 30$ & 119 & 122 & 584 \\ \hline
    $30 \times 0$ & 4 & 7 & 21\\ \hline
    $65535 \times 65535$ & 419 & 422 & 2049 \\ \hline
  \end{tabular}
\end{table}


\subsubsection*{剰余算}
中田:
実行サイクル数
\begin{table}[h]
  \begin{tabular}{|l|l|l|l|l|} \hline
    入力 & ステップ数(前回) & ステップ数(今回) & 実行サイクル数(今回) \\ \hline
    $30 \div 7$ & 362 & 365 & 1715 \\ \hline
    $0 \div 15$ & 355 & 358 & 1681 \\ \hline
    $65535 \div 1$ & 467 & 470 & 2225 \\ \hline
    $65535 \div 65535$ & 362 & 365 & 1715 \\ \hline
  \end{tabular}
\end{table}

※0÷15の前回のステップ数に誤りがありましたので訂正しました。\\


\subsubsection*{16進$\rightarrow$ 10進}
中田:
実行サイクル数
\begin{table}[h]
  \begin{tabular}{|l|l|l|l|l|} \hline
    入力 & ステップ数(前回) & ステップ数(今回) & 実行サイクル数(今回) \\ \hline
    $000F$ & 487 & 490 & 2387 \\ \hline
    $FFFF$ & 1724 & 1727 & 8292 \\ \hline
    $00XX$ & 74 & 30877 & 31212 \\ \hline
    $ABCX$ & 107 & 30910 & 31391 \\ \hline
  \end{tabular}
\end{table}

\subsubsection*{素数計算}
中田:
実行サイクル数
\begin{table}[h]
  \begin{tabular}{|l|l|l|l|l|} \hline
    入力 & ステップ数(前回) & ステップ数(今回) & 実行サイクル数(今回) \\ \hline
    $2$ & 378 & 381 & 1795 \\ \hline
    $64$ & 24240 & 24243 & 115153 \\ \hline
    $255$ & 100047 & 100050 & 475274 \\ \hline
    $65535$ & 24769799 & 24769801\footnotemark[1] & 117659446 \footnotemark[1] \\ \hline
  \end{tabular}
\end{table}
\footnotetext[1]{CPU\_MONITORINGを消したため、他のものとくらべ1少ない}

\subsubsection*{結果について}
中田: \\
16進数の変換における出力がエラーとなる演算とCPU\_MONITORINGをコメントアウトした演算を除き、 \\
全ての演算でステップ数が3増えた。これは、全ての演算において演算終了後に同じ命令を3回繰り返してしまうのが原因である。 \\
また、16進数の変換における出力がエラーとなる演算では、演算終了後に同じ命令が繰り返され、verilogでのステップ数が膨大になってしまった。

\subsubsection*{考察}
中田: \\

\subsubsection*{ステップ数と実行サイクル数について}
ステップ数は実行された命令の数であり、実行サイクル数は実行された命令の命令実行サイクルを \\
全て足し合わせた総数になるはずである。 \\
1つの命令に対し命令実行サイクルは4~7であるが、実際、正常に終了した演算の命令実行サイクルを \\
ステップ数で割った値の平均は4.83となったため、実験結果は妥当であったと言える。 \\
\\
\subsubsection*{最後に3回同じ命令が繰り返されてしまう点について}
test\_fpga\_ex3.vではステップ数と実行サイクル数のカウントをタスクSHOW\_CPU\_STATUSを呼び出すことで行っている。 \\
このSHOW\_CPU\_STATUSは呼び出された時にクロックが落ち、EX3が実行状態であると実行されるタスクである。 \\
この問題は、このSHOW\_CPU\_STATUSを余分に3回呼び出しているために発生していると考える。 \\
以下、その3つの実行サイクルについて考察する。 \\
\\
・1つ目の実行サイクルについて \\
これは次のwhile文内でSHOW\_CPU\_STATUSを呼び出しているためであると考える。 \\

\begin{lstlisting}[caption=test\_fpga\_ex3.v 264行目]
while (s | (|(~FPGA_EX3.fgo))) SHOW_CPU_STATUS(0);
\end{lstlisting}

このwhile文は出力フラグレジスタが初期状態の3であれば、s=1の間は繰り返され、HLTによってs=0となるとループが終了する。 \\
しかし、実際は最後の実行サイクルにあたるSHOW\_CPU\_STATUSが実行された後、クロックが立ちHLTによってsが0に切り替わる前に
while文の条件式の判定が実行されてしまい、次のクロックが落ちるときに余分にSHOW\_CPU\_STATUSが実行されてしまうのである。これによって1回余分に実行サイクルが生まれていると考えられる。 \\
\\
・2つ目の実行サイクルについて \\
これは次の文によって必ず発生する。ここでは引数1をとっているが、引数が1の場合はCPU\_MONITORINGのコメントアウトの有無に関わらず、内部の状態がターミナルに表示される。そのため、CPU\_MONITORINGをコメントアウトした場合はここでの内部の状態が表示され、実行サイクル数が2つ増えたものとなる。

\begin{lstlisting}[caption=test\_fpga\_ex3.v 265行目]
SHOW_CPU_STATUS(1);
\end{lstlisting}

・3つ目の実行サイクルについて
これは次のタスクKEY\_ACTION[1]内の1つ目のSHOW\_CPU\_STATUSで発生していると考えられる。 \\

\begin{lstlisting}[caption=test\_fpga\_ex3.v 268行目]
KEY_ACTION(1); /// run state --> stop state
\end{lstlisting}

\begin{lstlisting}[caption=test\_fpga\_ex3.v 225行目]
SHOW_CPU_STATUS(0); KEY[key_idx] <= 0;
\end{lstlisting}

1つ目のSHOW\_CPU\_STATUSはEX3が実行状態であるため実行されるが、次にKEY[1]←0が実行されEX3が中断状態になり、それ以降にSHOW\_CPU\_STATUSを呼び出しても実行されないのである。 \\

・3つの実行されている実行サイクルについて \\
HLT命令以降はs=0となり、scが常にクリアされ0になるため、1つ目の実行サイクルのAR←PCが常に実行されていると考えられる。 \\

\subsubsection*{エラー出力となる演算において、実行サイクル数が膨大になる点について}
これはエラーの出力にシリアルポートを利用しているため出力に時間がかかり、その間出力フラグレジスタが1となり
ソースコード9のwhile文が繰り返されていることが原因だと考えられる。 \\
シリアルポート通信に時間がかかってしまう点については、シミュレーションレポート2と内容が被るため、ここでは省略する。 \\

\newpage

\subsection*{2. シリアルポートを用いたtest\_io1およびパラレルポートを用いたものに変更したプログラムのシミュレーション時間の違いについての考察}



\subsubsection*{test\_io1:シリアルポート}
実行結果
\begin{table}[h]
  \begin{tabular}{|l|l|l|l|l|} \hline
    入力 & ステップ数 & 実行サイクル数 & シミュレーション時間 \\ \hline
    $ctrl-D$ & 37072 & 61991 & 6235000 \\ \hline
    $a→ctrl-D$ & 43281 & 92992 & 9335100 \\ \hline
    $a→b→ctrl-D$ & 49475 & 123990 & 12434900 \\ \hline
  \end{tabular}
\end{table}

\subsubsection*{test\_io1:パラレルポート}
実行結果
\begin{table}[h]
  \begin{tabular}{|l|l|l|l|l|} \hline
    入力 & ステップ数 & 実行サイクル数 & シミュレーション時間 \\ \hline
    $ctrl-D$ & 73 & 344 & 70300 \\ \hline
    $a→ctrl-D$ & 123 & 590 & 94900 \\ \hline
    $a→b→ctrl-D$ & 161 & 776 & 113500 \\ \hline
  \end{tabular}
\end{table}
\\
\\
\\
UART-RXおよびUART-TXの通信開始から終了までのシミュレーション時間: 3093200

\subsubsection*{考察}

\subsubsection*{クロックとシミュレーション時間について}
以下の記述からクロックはシミュレーション時間で50ごとに切り替わる。よってシミュレーション時間での100が
クロックの1周期であり、シミュレーション時間を100で割ったものがクロックの発生回数である。 \\

\begin{lstlisting}[caption=test\_fpga\_ex3.v 92行目]
always # 50 clk = ~clk;
\end{lstlisting}

\subsubsection*{シミュレーション時間の違いについて}
実験結果よりシリアルポートを使用した方がシミュレーション時間がパラレルポートに比べ100倍近く増えていることがわかる。
これは、パラレルポートでは入出力に1クロック分のシミュレーション時間がかかっているのに対し、シリアルポートではUARTの通信を行うごとにシミュレーション時間が3093200かかっているためであると考えられる。 \\
シリアルポートにおいて、入力が1つの場合はUART-RX、UART-TXそれそれ一回ずつ通信を行うため、全てのUART通信にかかるシミュレーション時間は3093200×2 =6186400となる。 \\
複数の入力の場合は、2つ目以降のUART-RX通信ではその一つ前の入力のUART-TXとほぼ並列に通信が行われるため、
k個の入力があった場合、全てのUART通信にかかるシミュレーション時間はおよそ3093200×(k+1)となると言える。
test\_io1では入出力以外の命令にかかるシミュレーション時間は3093200に比べ極めて少ないため、実際に実行結果でも
入力数をk個とするとシミュレーション時間が3093200×(k+1)に近い値をとっている。 \\

\subsubsection*{UART通信にかかるシミュレーション時間について}
・UART-RXについて \\
uart\_rxモジュールは1bitごとの入力を受け取り、8bitのデータとして返すモジュールである。この動作原理を説明する。 \\
uart\_rxでは11bit(start bit 1bit + 入力データ 8bit + parity bit 1bit + end bit 1bit)のレジスタrx\_shiftが用意されており、
そこに右シフトによって1bitごとにビットを格納していき、全てのシフトが完了するとrx\_shift(8:1)を出力する。 \\
この右シフトはボーフェイズごとに行われ、ボーフェイズは4つのフェイズに分かれている。1つのフェイズではクロックの発生を
baud\_tickによって702回カウントした後、フェイズごとに違った動作を行う。フェイズ1、フェイズ3ではカウント後にbaud\_tickを0にリセットし次のフェイズに移行する。フェイズ2ではカウント後、rx\_shiftの右シフトを行い、baud\_tickを0にリセットして次のフェイズに移行する。フェイズ4ではカウント後、baud\_tickを0にリセットし、bit\_countを1増やして次のボーフェイズのフェイズ1に移行する。 \\
この一連のフェイズを1つのボーフェイズとして、このボーフェイズが11回繰り返される。 \\
つまり、1つのボーフェイズでは(702+1)×4=2812回クロックが発生することになる。これは1bit読み込むのに必要なクロック数であり、動作周波数をボーレートで割った値 27MHz÷9.6KHz=2812.5に近い値となっている。値に誤差がある原因は、BAUD\_PERIODの値を小数点以下切り捨てにして定めているためである。
また、ボーフェイズは11回繰り返されるため、1回のUART-RX通信でクロックは2812×11=30932回発生することになる。これは実行結果のUART-RX通信の開始から終了までにクロックが発生した回数に等しい。 \\
\\
\\
・UART-TXについて \\
uart\_txモジュールは8bitの入力を受け取り、1bitごとのデータを返すモジュールである。この動作原理について説明する。 \\
uart\_txにもuart\_rxと同様に11bitのレジスタtx\_shiftが用意されており、tx\_shiftは\{1,tx\_parity,入力データ 8bit,0\}と初期化される。出力ポートのtx\_doutはtx\_shift[0]が連続代入されており、tx\_doutの値はtx\_shiftを右シフトするごとに切り替わる。この右シフトを11回繰り返すことで、1bitごとの出力を行っている。 \\
uart\_txでもボーフェイズが存在するが4つのフェイズには分かれていない。uart\_txでは1つのボーフェイズ内でbaud\_tickによってクロックの発生を2811回カウントし、次のクロックでtx\_shiftを右シフトし、baud\_tickの値を0にリセットする。このボーフェイズを11回繰り返している。 \\
uart\_txのボーフェイズでもクロックが2812回発生し、一回のUART-TX通信でクロックは2812×11=30932回発生する。これについても実行結果に一致している。





\end{document}
