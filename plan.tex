\documentclass{jsarticle}
\usepackage{graphicx,listings}
\lstset{language={C},%
  basicstyle=\footnotesize,%
  commentstyle=\textit,%
  classoffset=1,%
  keywordstyle=\bfseries,%
  frame=tRBl,framesep=5pt,%
  showstringspaces=false,%
  numbers=left,stepnumber=1,numberstyle=\footnotesize%
}%


\title{情報実験第三 実験計画書}

\author{情報工学科15\_03602 柿沼 建太郎 \\ 情報工学科 15\_10588 中田 光}
\date{\today}
\begin{document}
\maketitle

\section*{概要}
担当者は15班の柿沼・中田の2名である。
今回の実験では本班はtest\_calc1.asmの改良を目標とする。
なお、課題1-Cで行った7セグメントディスプレイへの表示は行わない。

\section*{追加/変更する機能}
\subsection*{Verilogコードの追加/変更}
\begin{itemize}
    \item 16bitCPUを32bitCPUに変える
    \item オペコードを1bit増やし、命令数を増やす
    \item ex3の命令に乗算/除算を実装
    \item 固定小数を実装
    \item メモリ参照命令を追加(減算,乗算,除算,剰余)
\end{itemize}
\subsection*{EX3コードの追加/変更}
\begin{itemize}
    \item 小数の入力受付
    \item 表示の10進化
    \item 乗算,除算,剰余,平方根などの新たな演算の実装
\end{itemize}

\section*{担当者}
各タスクと担当者を表として以下に示す。
\begin{table}[h]
\begin{tabular}{|l|c|c|} \hline
タスク/名前 & 柿沼 & 中田 \\ \hline \hline
Verilogコード変更 &  & $\circ$ \\ \hline
EX3コード変更 & $\circ$ & \\ \hline
\end{tabular}
\end{table}

\section*{スケジュール}
\subsection*{Verilog}
\begin{table}[h]
\begin{tabular}{|l|c|c|} \hline
    日付 & 概要 \\ \hline \hline
7/7 & CPUの32bit化 \\ \hline
7/10 & 命令セットの変更 \\ \hline
7/14 & 命令セットの変更 \\ \hline
7/21 & 固定小数の実装 \\ \hline
7/24 & 固定小数の実装 \\ \hline
7/28 & 予備日 \\ \hline
\end{tabular}
\end{table}

\subsection*{EX3}
\begin{table}[h]
\begin{tabular}{|l|c|c|} \hline
    日付 & 概要 \\ \hline \hline
7/7 & 表示を10進に直す \\ \hline
7/10 & 小数の入力受付 \\ \hline
7/14 & 小数の入力受付 \\ \hline
7/21 & 各種演算の実装 \\ \hline
7/24 & 各種演算の実装 \\ \hline
7/28 & 予備日 \\ \hline
\end{tabular}
\end{table}

\end{document}
